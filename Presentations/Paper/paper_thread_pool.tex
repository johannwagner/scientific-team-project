\documentclass[conference]{IEEEtran}
\IEEEoverridecommandlockouts
% The preceding line is only needed to identify funding in the first footnote. If that is unneeded, please comment it out.
\usepackage{cite}
\usepackage{amsmath,amssymb,amsfonts}
\usepackage{multicol}
\usepackage{algorithmic}
\usepackage{graphicx}
\usepackage{textcomp}
\def\BibTeX{{\rm B\kern-.05em{\sc i\kern-.025em b}\kern-.08em
    T\kern-.1667em\lower.7ex\hbox{E}\kern-.125emX}}
\begin{document}

\title{Thread Pool in GeckoDB/BOLSTER - Integrating a Thread Pool into a high performance Database System \\
}

\author{\IEEEauthorblockN{Marten Wallewein-Eising}
	\IEEEauthorblockA{\textit{Otto-von-Guericke University} \\
	Magdeburg, Germany \\
	marten.wallewein-eising@st.ovgu.de}\and
	\IEEEauthorblockN{Robert Jendersie}
	\IEEEauthorblockA{\textit{Otto-von-Guericke University} \\
		Magdeburg, Germany \\
		robert.jendersie@ovgu.de}\and
	\IEEEauthorblockN{Johannes Wuensche}
	\IEEEauthorblockA{\textit{Otto-von-Guericke University} \\
	Magdeburg, Germany \\
	johannes.wuensche@st.ovgu.de}\and
	\IEEEauthorblockN{Johann Wagner}
	\IEEEauthorblockA{\textit{Otto-von-Guericke University} \\
	Magdeburg, Germany \\
	johann.wagner@st.ovgu.de}
}


\maketitle

\begin{abstract}

\end{abstract}

\begin{IEEEkeywords}
Thread Pool, GeckoDB, BOLSTER
\end{IEEEkeywords}

\section{Introduction}
Since the amount of data that is stored and processed by modern database systems is growing fast, sequential data processing as only possibility is inconceivable. Applications have to process data in parallel to reach sufficient throughput to fulfil appropriate requirements. 

Parallel data processing can be achieved by different approaches like instruction and data parallelism or multi threading. In this paper, we focus on multi threading by implementing a thread pool for the graph database system GeckoDB. The thread pool will be integrated into BOLSTER, a high performance library for parallel execution of primitives like for or filter on large data sets. In the current implementation, BOLSTER creates a fix number of threads for each call of a primitive. This approach is called \emph{thread-per-request}. Since many primitives are executed at the same time, many drawbacks arise from this implementation. 
First of all, the creation of threads comes along with overhead like stack initialisation and memory allocation. Secondly, creating a huge number of threads simultaneously may lead to large context switch overhead of the scheduler. Additionally, debugging and profiling applications that create many threads during runtime is 	time-consuming.

To overcome these drawbacks, we integrate an optimised thread pool in BOLSTER. Along with the implementation, we measure the performance of the primitives to determine the thread pool overhead. Additionally, we measure metrics like \emph{idle and job time} of threads to evaluate correct thread pool sizes for the considered use cases.  In this work we make the following contributions:
\begin{itemize}
	\item We describe our design and implementation of the thread pool 
	\item We evaluate the possibility to wait for a group of task in the calling thread
	\item We compare our thread pool against the existing implementation in BOLSTER
\end{itemize}
We organized the rest of the paper as follows. In Section 2, we give preliminaries about the considered task configuration and about thread safe access of memory. In Section 3, we show our design and implementation of the thread pool and examine our experimental environment in Section 4. In Section 5, we describe the results of our performance evaluation. In Section 6, we name related work and state our conclusion and future work in Section 7.

\section{Preliminaries}
In this section, we define our configuration of tasks that are processed by the thread pool and state difficulties of synchronizing thread access to memory.

\subsection{Task Configuration}
We define a Task as a structure containing data that has to be processed and an operation that has to be executed on the data. In this work, we define tasks as \emph{independent}, which means tasks do not have dependencies on other tasks and can be processed independently. Furthermore, we expect the data passed to two tasks are stored in different memory locations. Consequently, while executing the task operation, threads do not access the same memory locations. Additionally, we only consider not preemtable tasks. Once a task is assigned to a thread, the thread will finish the operation of the task before getting a new one. 

\subsection{Synchronizing Memory Access from Threads}
Parallelism with multi threading works great as long as each thread works on a separate memory area. During task scheduling, the scheduler has to know the state each thread has. This can be solved with signal handling or by writing the current thread state into main memory. In this work, we implement the second approach to avoid having another thread that only schedules tasks to other threads. 

Threadsafe access to memory can be achieved by different approaches. Firstly, a mutex thread can manage the access for multiple other threads. Secondly, atomic operations ensure the safe access from different threads to the same memory. Since using a mutex thread can decrease access performance, we decided to execute atomic operations on the memory containing the thread state informations. For example, the function \emph{atomic\_compare\_exchange\_strong} performs a compare and, if the result is true, an exchange of the memory in one atomic operation.

%\begin{figure}
%	\includegraphics[width=0.4\textwidth]{figure_1.png}
%	\caption{Coherence between data, instructions, and the results of SISD and SIMD.}
%	\label{fig}
%\end{figure}
% SIMD explanation

%\begin{table*}[htbp]
%	\caption{SIMD instructions from Streaming SIMD Extensions 2 (SSE2)}
%	\begin{center}
%		\begin{tabular}{|c|c|}
%			\hline
%			\textbf{SIMD instruction}&\textbf{Explanation}\\
%			\hline
%			\_\_m128i \_mm\_load\_si128 (\_\_m128i *p) & Loads a 128-bit value. Returns the value loaded into a variable representing a register.\\
%			\_\_m128i \_mm\_cmpgt\_epi32 (\_\_m128i a, \_\_m128i b) & Compares 4 signed 32-bit integers in a and 4 signed 32-bit integers
%			in b for greater-than.\\
%			\_\_m128i \_mm\_set1\_epi32(int i) & Sets the four signed 32-bit integer values to i.\\
%			\hline
%		\end{tabular}
%		\label{tab1}
%	\end{center}
%
%\end{table*}

\section{Implementation}
In this section, 

\section{Experimental Environment}

\section{Evaluation}

\section{Related Work}

\section{Conclusion and Future Work}

\bibliographystyle{splncs}
\bibliography{paper_thread_pool} 

\end{document}
